\documentclass[10pt,a4]{article}
% \topmargin-1.0cm	% Normal margin
\topmargin-0.75in	% Narrow margin
\advance\oddsidemargin-2cm
\advance\evensidemargin-2cm
% \textheight8.5in	% Normal margin
\textheight9.25in	% Narrow margin
\textwidth6.4in
\newcommand\bb[1]{\mbox{\em #1}}
\def\baselinestretch{0.9}

\usepackage{multicol}
% The use of the times package forces the use of the type-1 times
% roman font, but the times roman font does not look nice.
% Besides the times roman font still does not print correctly on
% the dopy printer.
%\usepackage{times}

\usepackage{fancyhdr}
\usepackage{origpagecounting}
\usepackage[dvips]{color}
\usepackage[colorlinks]{hyperref}

\newcounter{myEnumCounter}
\newcounter{mySaveCounter}
\renewenvironment{enumerate}{%
  \begin{list}{\arabic{myEnumCounter}.}{\usecounter{myEnumCounter}%
  \setcounter{myEnumCounter}{\value{mySaveCounter}}}
  }{%
  \setcounter{mySaveCounter}{\value{myEnumCounter}}\end{list}%
}
\newcommand\myEnumReset{\setcounter{mySaveCounter}{0}}

\definecolor{gray}{rgb}{0.4,0.4,0.4}

\begin{document}

\thispagestyle{empty}
%\pagestyle{plain}

\pagestyle{fancy}
\fancyhf{}
\cfoot{{\thepage}}
\renewcommand{\headrulewidth}{0pt}
\renewcommand{\footrulewidth}{0pt}
\long\def\ignore#1{}
\sloppypar

%\fancyfoot[C]{\footnotesize \textcolor{gray}{}}


\begin{center}
\hspace{-0.4in}{\huge \bf Mohamed Assem Ibrahim}
\vspace*{0.5cm}
\end{center}

\begin{tabbing}
\=xxxxxxxx\=xxxxxxxx\=xxxxxxxx\=\kill
\begin{tabular*}{\linewidth}{l@{\extracolsep{\fill}}l}

McGlothlin-Street Hall, 101B  & Email: maibrahim@email.wm.edu \\
Williamsburg, VA 23187 &  Contact Number: +1-757-604-9355 \\
% Fax Number: +1-757-221-1717 & Homepage: \url{http://massemibrahim.github.io/}    \\
& Homepage: \url{http://massemibrahim.github.io/}    \\
\end{tabular*}
\end{tabbing}

\vspace*{0.2cm}

%==========================================
%\vspace{0.20cm}
\subsection*{RESEARCH INTERESTS}
\hrule
\vspace{0.2cm}
\begin{list}{}{}
\item 
My research interests lie in all aspects of computer architecture, including data-parallel architectures (e.g., GPUs), CPU-GPU heterogeneous architectures, and interconnection networks.
\end{list}

\subsection*{EDUCATION}

\hrule
\vspace{0.2cm}

% \begin{tabbing}
% xxxx\=xxxxxxxx\=xxxxxxxx\=xxxxxxxx\=\kill
% %\>\begin{tabular*}{6.1in}{lr}

% \>\begin{tabular*}{0.9\linewidth}{l@{\extracolsep{\fill}}l}
% {\bf College of William and Mary}, VA & {\it Spring 2016 - Present} \\
% {\it Ph.D. Candidate} in Computer Science Department\\
% {\it Current GPA} = 4
% & \\
% & \\
% {\bf Cairo University}, Giza, Egypt & {\it Fall 2010 - Fall 2015} \\
% Masters of Science {\it(M.Sc.)} in Computer Engineering \\
% {\it GPA} = 4
% & \\
% & \\
% {\bf Cairo University}, Giza, Egypt & {\it Fall 2005 - Spring 2010} \\
% Bachelor of Science {\it(B.Sc.)} {\it (Hons.)} in Computer Engineering \\
% {\it GPA} = 3.91 (Calculated with \href{http://www.foreigncredits.com/Resources/GPA-Calculator/}{Foreign Credits GPA Calculator}) 
% \end{tabular*}
% \end{tabbing}

\begin{itemize}

\item{\bf College of William and Mary (W\&M) \hfill {\bf Spring 2016 -- Present}} \\
% {\bf Williamsburg, VA}\\
{\it Ph.D. Candidate} in Computer Science Department \hfill {\bf Williamsburg, VA}\\
{\it GPA} = 4

\item{\bf Cairo University (CU) \hfill {\bf Fall 2010 -- Fall 2015}} \\
% {\bf Giza, Egypt}\\
% {\it Masters of Science (M.Sc.)} in Computer Engineering \hfill {\bf Giza, Egypt}\\
{\it M.Sc.} in Computer Engineering \hfill {\bf Giza, Egypt}\\
{\it GPA} = 4

\item{\bf Cairo University (CU) \hfill {\bf Fall 2005 -- Spring 2010}} \\
% {\bf Giza, Egypt}\\
% {\it Bachelor of Science (B.Sc.) (Hons.)} in Computer Engineering \hfill {\bf Giza, Egypt}\\
{\it B.Sc. (Hons.)} in Computer Engineering \hfill {\bf Giza, Egypt}\\
{\it GPA} = 3.91 (Calculated with \href{http://www.foreigncredits.com/Resources/GPA-Calculator/}{Foreign Credits GPA Calculator})

\end{itemize}

\subsection*{PROFESSIONAL EXPERIENCE}
\hrule
\vspace{0.2cm}
\begin{itemize}

\item{\bf College of William and Mary (W\&M) \hfill {\bf January 2016 -- Present}}\\
{\it Research Assistant} \hfill {\bf Williamsburg, VA}\\
{\it Advisor:} Assistant Professor \href{http://www.cs.wm.edu/~adwait/}{Adwait Jog} \\
My role is to conduct research related to Large-Scale GPU architectures. 

\item{\bf Cairo University (CU) \hfill {\bf August 2010 -- December 2015}}\\
{\it Research Assistant} \hfill {\bf Giza, Egypt}\\
{\it Advisor:} Professor \href{http://scholar.cu.edu.eg/?q=helboghdadi}{Hatem El-Boghdadi}\\
My role was to conduct research related to Bufferless Network-on-Chip.

\item{\bf Nile University (NU) \hfill {\bf June 2012 -- June 2013}}\\
{\it Research Software Development Engineer (RSDE)} \hfill {\bf Giza, Egypt}\\
{\it Advisor:} Associate Professor \href{https://sites.google.com/site/telbatt/}{Tamer ElBatt}\\
My role was to conduct research and create functional prototypes related to predictive loading of content on mobile phones based on user modeling. The results of this work are published in ICC 2014 and MobiSys 2013. 

\item{\bf Inmobly \hfill {\bf October 2011 -- June 2013}}\\
{\it Software Engineer} \hfill {\bf Giza, Egypt}\\
{\it Manager:} Professor \href{https://ece.osu.edu/people/elgamal}{Hesham ElGamal}\\ 
{\it Mentor:} Associate Professor \href{https://eg.linkedin.com/in/nwanas}{Nayer Wanas}\\
% {\it Projects:} PAUL, inFootball, Agenda25\\
My role was to implement some of the required functionalities for several mobile applications. In {\it PAUL}, I implemented logging and reporting modules (for Wi­Fi, content usage and battery), and scripts to analyze the logs and to build the user profile to correctly predict the different users' behavior. In {\it inFootball}, I implemented the news crawler and the handlers for the different client-side requests. In {\it Agenda25}, I implemented the BlackBerry version of the application.

% Worked on PAUL - Predictive Automated User-Centric Loading technology that offers an innovative solution for the bandwidth crunch problems facing wireless operators. PAUL utilizes the predictability of user behavior to offload the peak traffic on cellular network by utilizing non-peak intervals and/or WI-FI networks. 
% I was mainly involved in back-end logic; developing scheduling algorithms responsible for offloading the peak traffic. Many challenges were considered like: 
% - For each user which content will be rerouted and to which interval,
% - which intervals to choose to maximize content freshness,
% - uncertainty about which intervals will be available,
% - the available Bandwidth in each,
% - uncertainty about the content to be consumed by the user

% $\bullet$ Proposed various techniques to improve the performance and energy efficiency of GPU hardware.
% Researched on efficient execution of multiple contexts/applications on next generation GPUs.
% The results of this work are published in GPGPU 2014 (co-located with ASPLOS 2014).
% Implemented and evaluated micro-architecture techniques for Intel's ultra-low power core ({\it Siskiyou}).
% This infrastructure is released to universities to perform research on energy-efficient architectures.

\end{itemize}

\subsection*{AWARDS, GRANTS, and HONORS}
\hrule
\vspace{0.2cm}
\begin{itemize}
\item Graudate Assistantship, College of William and Mary, 2016
\item Student Travel Grant for attending: ISCA 2015, HPCA 2017, MICRO 2017
\item Graduate Student Association Conference Funds, College of William and Mary, 2017
\item Graudate Assistantship, Cairo University, 2010
\item Best Graduation Project (Software Engineering Category), Egyptian Engineering Day (EED), IEEE GOLD, 2010
\end{itemize}

\subsection*{PUBLICATIONS}
\hrule
\vspace{0.2cm}
% {\bf Note:} In my sub-discipline (computer hardware architecture) of computer science, conference publication 
% is preferred to journal publication, and the premier conferences are generally more selective than the premier journals. 
% The premier conferences in computer hardware architecture and related areas (not in any particular order)
% are: MICRO, ISCA, HPCA, ASPLOS, PACT, SIGMETRICS, DAC, DATE. The
% acceptance rate of all these premier conferences is around 20\%. 
% See \url{https://www.cs.ucsd.edu/sites/cse/files/cse/assets/docs/arch.pdf} for more details.  \\
 
% \underline{Underlined} students are advised by me. %(or were advised) 

\begin{description}
\item 
{\bf [HPCA 2018]}
Haonan Wang, Fan Luo, {\tt \underline{Mohamed Ibrahim}}, Onur Kayiran, Adwait Jog, 
{\it Efficient and Fair Multi-programming in GPUs via Pattern-based Effective Bandwidth Management},
In the Proceedings of $24^{th}$ International Symposium on High-Performance Computer Architecture (HPCA), 
Vienna, Austria, February, 2018
\textbf{\textit{Acceptance rate: 54/260 $\approx$ 20\%}}

\item 
{\bf [AIM 2017]}
Hengyu Zhao, Colin Weinshenker, {\tt \underline{Mohamed Ibrahim}}, Adwait Jog, Jishen Zhao,
{\it Layer-wise Performance Bottleneck Analysis of Deep Neural Networks},
In the Proceedings of $1^{st}$ International Workshop on Architectures for Intelligent Machine (AIM), Portland, Oregon, September, 2017

\item 
{\bf [HPCA 2017]}
Xulong Tang, Ashutosh Pattnaik, Huaipan Jiang, Onur Kayiran, Adwait Jog, Sreepathi Pai, {\tt \underline{Mohamed Ibrahim}}, Mahmut Kandemir, Chita Das, 
{\it Controlled Kernel Launch for Dynamic Parallelism in GPUs},
In the Proceedings of $23^{rd}$ International Symposium on High-Performance Computer Architecture (HPCA), 
Austin, Texas, February, 2017
\textbf{\textit{Acceptance rate: 50/224 $\approx$ 22\%}}

% \item 
% {\bf [M.Sc. Thesis 2015]}
% {\tt Mohamed Ibrahim}, {\it On Enhancing the Performance of Bufferless Network-on-Chip}, 
% M.Sc. Thesis, Cairo University, Giza, Egypt, 2015

\item 
{\bf [MES 2015]}
{\tt \underline{Mohamed Ibrahim}}, Hatem M El-Boghdadi,
{\it Investigating the Viability of Maximum Flexibility Selection Function in Bufferless 2D Meshes},
In the Proceedings of $3^{rd}$ International Workshop on Many-core Embedded Systems (MES), Portland, Oregon, June, 2015

\item{\bf [ICC 2014]} 
Omar Shoukry, {\tt \underline{Mohamed Ibrahim}}, John Tadrous, Hesham El Gamal, Tamer ElBatt, Nayer Wanas, Yaser Elnakieb, and Mohamed Khairy 
{\it Proactive Scheduling for Content Pre­fetching in Mobile Networks}, 
In the Proceedings of IEEE International Conference on Communications (ICC), Sydney, Australia, June, 2014
\textbf{\textit{Acceptance rate: 995/2,608 $\approx$ 38\%}} 

\item{\bf [MobiSys 2013]} 
{\tt \underline{Mohamed Ibrahim}}, Omar Shoukry, Hesham El Gamal, Tamer ElBatt, Nayer Wanas, Mohamed Abdel Raouf, Mohamed Zakaria, Ahmed Abdel Kader and Hakem Zayed 
{\it PAUL­ Proactive Automated mobile User centric content deLivery}, 
In the Proceedings of $11^{th}$ International Conference on Mobile Systems, Applications, and Services (MobiSys), Taipei, Taiwan, June, 2013

\end{description}

% \subsection*{PAPERS UNDER SUBMISSION}
% \hrule
% \vspace{0.2cm}

% \begin{description}
% \item 
% {\bf [HPCA 2018]}
% Haonan Wang, Fan Luo, {\tt \underline{Mohamed Ibrahim}}, Onur Kayiran, Adwait Jog, 
% {\it Efficient and Fair Multi-programming in GPUs via Pattern-based Effective Bandwidth Management}

% \item 
% {\bf [HPCA 2018]}
% Haonan Wang, {\tt \underline{Mohamed Ibrahim}}, Sparsh Mittal, Adwait Jog, 
% {\it ASAP: Address-Stride Assisted Approximate Value Predictor for GPUs}

% \end{description}

% \subsection*{PATENTS \& BOOK CHAPTERS}
% \hrule
% \vspace{0.2cm}

% \begin{description}

% \item{\bf [Book Chapter]}
% Nandita Vijaykumar, Gennady Pekhimenko, {\tt Adwait Jog}, Saugata Ghose, Abhishek Bhowmick, 
% Rachata Ausavarungnirun, Chita Das, Mahmut Kandemir, Todd C. Mowry, and Onur Mutlu, 
% A Framework for Accelerating Bottlenecks in GPU Execution with Assist Warps, 
% Book Chapter in Advances in GPU Research and Practice, Elsevier, to be published in 2016. 

% \item{\bf [US Patent]}
% Evgeny Bolotin, Zvika Guz, {\tt Adwait Jog}, Stephen W. Keckler, Mike Parker, 
% Approach to Adpative Allocation of Shared Resources in Computer Systems, 
% United States Patent Application US20150163324 A1 

% \end{description}

% \subsection*{STUDENTS CURRENTLY ADVISING}
% \hrule
% \vspace{0.2cm}
% \begin{itemize}
% \item {\bf Ph.D. Students:} Mohamed Ibrahim, Haonan Wang
% \item {\bf MS Students:}  Robert Risque 
% \item {\bf Undergraduate Students:}  Colin Weinshenker (Honors student)
% \end{itemize}

% \subsection*{STUDENTS GRADUATED}
% \hrule
% \vspace{0.2cm}
% \begin{itemize}
% \item {\bf MS Students:}  Fan Luo 
% \item {\bf Undergraduate Students:}  Robert Risque
% \end{itemize}

\subsection*{TECHNICAL STRENGTHS}
\hrule
\vspace{0.2cm}
\begin{itemize}
\item {\bf Programming} C/C++, Java, Python, OpenMP, MPI, MATLAB
\item {\bf Tools} GPGPU-Sim, gem5, SimpleScalar
\end{itemize}

\subsection*{TEACHING EXPERIENCE}
\hrule
\vspace{0.2cm}
\begin{itemize}
\item{\bf Teaching Assistant@W\&M}, CS 421, Database Systems \hfill {\bf Spring 2017} 
\item{\bf Teaching Assistant@W\&M}, CS 424/524, Computer Architecture \hfill {\bf Fall 2016} 
\item{\bf Teaching Assistant@W\&M}, CS 421, Database Systems \hfill {\bf Spring 2016} 
\item{\bf Teaching Assistant@W\&M}, CS 131, Concepts of Computer Science \hfill {\bf Spring 2016}

% {\bf Note:} \href{teaching.html}{Here} is the (almost) complete list of courses.

\end{itemize}

\subsection*{REFERENCES}
\hrule
\vspace{0.2cm}
\begin{itemize}
\item 
{\bf Adwait Jog}\\
McGlothlin-Street Hall 111, College of William and Mary\\
{\bf Email:} adwait@cs.wm.edu\\
{\bf Contact Number:} +1-757-221-1434
\end{itemize}

% \subsection*{INVITED TALKS}
% \hrule
% \vspace{0.2cm}
% \begin{itemize}
% \item Breaking the Memory Bandwidth Wall in GPUs  \\
% -- Virginia Commonwealth University (VCU), Feb 2016 \\
% -- Indian Institute of Science, Bangalore, India, Dec 2015

% \item Anatomy of GPU Memory System for Multi-Application Execution, \\
% -- MEMSYS 2015, Washington, DC, Oct 2015 

% \item The Future of Parallel Computing with GPUs \\
% -- The College of William and Mary, Feb 2015 \\
% -- University of Utah, Mar 2015 \\
% -- Temple University, Mar 2015 \\
% -- AMD Research, Mar 2015 \\
% -- UC Riverside, Apr 2015 \\
% -- Intel Labs, Apr 2015 

% \item Application-aware Memory System for Fair and Efficient Execution of Concurrent GPU Applications, \\
% -- GPGPU-7 Workshop (co-located with ASPLOS 2014), Salt Lake City, UT, March 2014 \\
% -- Intern Talk, NVIDIA Research, Santa Clara, CA, Sept 2013

% \item Mitigating and Masking the Limitations of GPU Memory Systems, \\
% -- Intern Talk, NVIDIA Research, Santa Clara, CA, June 2013

% \item Orchestrated Scheduling and Prefetching for GPGPUs, \\
% -- ISCA 2013, Tel Aviv, Israel, June 2013

% \item OWL: Cooperative Thread Array Aware Scheduling Techniques for Improving GPGPU performance, \\
% -- ASPLOS 2013, Houston, TX, March 2013 

% \item Cache Revive: Architecting Volatile STT-RAM Caches for Enhanced Performance in CMPs, \\
% -- DAC 2012, San Francisco, CA, June 2012\\
% -- Poster presentation at IUCRC NEXYS Workshop, Pittsburgh, PA
% \end{itemize}

% \subsection*{SERVICE AT W\&M}
% \hrule
% \vspace{0.2cm}
% \begin{itemize}
% 	\item Pre-major advising, Aug 2016 - Present
% 	\item Departmental Colloquium Committee, Aug 2016 - Present
% 	\item Departmental Admissions Committee, Aug 2015 - July 2016
% \end{itemize}

% \subsection*{EXTERNAL PROFESSIONAL SERVICE AND MEMBERSHIPS}
% \hrule
% \vspace{0.2cm}
% \begin{itemize}
% 	% \item Program Committee Member, HPDC 2017
% 	% \item Program Committee Member, DSN 2017
% 	% \item Program Committee Member, ICPADS 2016 
% 	% \item Program Committee Member, ICS 2016
% 	% \item Program Committee Member, NAS 2016
% 	% \item Program Committee Member, ICPP 2016
% 	% \item Program Committee Member, GPGPU (2017, 2016)
% 	% \item Proceedings and Submission Chair, ANCS 2015
% 	% \item Invited Reviewer (Journals):  \\
% 	% -- ACM Transactions on Architecture and Code Optimization (TACO) \\
% 	% -- IEEE Transactions on Computers (TC) \\ 
% 	% -- IEEE Journal on Computer Architecture Letters (CAL) \\
% 	% -- ACM Transactions on Parallel Computing (TOPC) \\
% 	% -- ACM Transactions on Embedded Computing (TECS) \\
% 	% -- IEEE Transactions on Parallel and Distributed Systems (TPDS) \\
% 	% -- ACM Transactions on Design Automation of Electronic Systems (TODAES) \\
% 	% \item Invited External Reviewer (Conferences):
% 	% DAC 2013, HPCA 2013, MICRO 2012, and ICCD (2014, 2013)
% 	\item Member of ACM, IEEE, ACM SIGARCH
% \end{itemize}

\end{document}